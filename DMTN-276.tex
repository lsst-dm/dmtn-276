\documentclass[DM,authoryear,toc]{lsstdoc}
\input{meta}
\graphicspath{{./}{figures/}}
\usepackage{amsmath}
% Package imports go here.

% Local commands go here.

%If you want glossaries
%\input{aglossary.tex}
%\makeglossaries

\title{Effects of Persistence on E2V Sensors and its Impacts on DC2 Data}

% Optional subtitle
% \setDocSubtitle{A subtitle}

\author{%
John Banovetz,
Yousuke Utsumi,
Colin Slater
}

\setDocRef{DMTN-276}
\setDocUpstreamLocation{\url{https://github.com/lsst-dm/dmtn-276}}

\date{\vcsDate}

% Optional: name of the document's curator
% \setDocCurator{The Curator of this Document}

\setDocAbstract{%
In this technote, I describe the measurements and characterization of the persistence effect with the E2V sensors. 
I will also describe the process that I used to evaluate the effects persistence would have on the LSST survey. 
For this, I took persistence characterization values from electro-optical testing and modified raw DC2 images to include this model of persistence. 
I then ran the raw images through the DRP and analyzed the results. 
I found that the persistence adds a possibly non-negligible amount of flux to faint sources and can affect a significant amount of pixels. 
}

% Change history defined here.
% Order: oldest first.
% Fields: VERSION, DATE, DESCRIPTION, OWNER NAME.
% See LPM-51 for version number policy.
\setDocChangeRecord{%
  \addtohist{1}{YYYY-MM-DD}{Unreleased.}{John Banovetz}
}


\begin{document}

% Create the title page.
\maketitle
% Frequently for a technote we do not want a title page  uncomment this to remove the title page and changelog.
% use \mkshorttitle to remove the extra pages

% ADD CONTENT HERE
% You can also use the \input command to include several content files.

\section{Electro-optical Measured Persistence}
% \begin{figure*}[!htp]
%   \centering
%   \includegraphics[width=0.95\textwidth, angle=0]{Run_5_persistence_ex.png}
%   \caption{
%   Example of peristence in Run 5 data. 
%   This figure shows the initial flash on the left most image (dataId shown as the title) and the subsquent darks images going from left to right.
%   This particular flash has one of the three spots inbetween two amplifiers and you can see the trail starting at the spots and going all the way to the readout of each amplifier.
%   }\label{fig:ex_persistence_Run5}
% \end{figure*}

\subsection{Definition and Run 5 Measurements}

\begin{figure*}[!htp]
  \centering
  \includegraphics[width=0.95\textwidth, angle=0]{Run_5_persistence_ex_2.png}
  \caption{
  % Example of persistence in Run 5 data. 
  This figure shows the initial flash on the left most image (dataId shown as the title) and a portion of the subsquent darks images going from left to right.
  This particular flash has one of the three spots in between two amplifiers and you can see the trail starting at the spots and going all the way to the readout of each amplifier.
  }\label{fig:ex_persistence_Run5}
\end{figure*}

\begin{figure*}[!htp]
  \centering
  \includegraphics[width=0.95\textwidth, angle=0]{Persistence_Cuts.png}
  \caption{
  Horizontal and vertical cutouts of an example persistence spot (left) in a 2x2 binned image to highlight the effect. 
  This represents both the persistence in the spot (blue/red) and the trail (orange/red) as compared to the background (brown/purple).
  }\label{fig:cuts}
\end{figure*}

Persistence is a sensor level effect where charge gets trapped from one image and `persists' into the next at some fraction of the original charge. 
Figure \ref{fig:ex_persistence_Run5} shows an example of this from Run 5 data where three saturated spots ($>10**5$ e-/pixel) persist into subsequent dark images. 
Persistence for LSSTCam was first found in Run 5 of electro-optical (EO) testing at SLAC from crosstalk measurements. 
Further investigation found that this effect only affected E2V sensors
% This persistence was found to have an average signal after the flash of 6 ADU and had a decay constant of 37 seconds. 
and that the persistence was only mitigated via integration time. 
Mitigation tests were done and found that while darks were able to release the trapped persistence charge, there were no signs of persistence in biases. 
One of the troublesome features of the persistence in the E2V sensors is the persistence trail. 
% Not only do the saturated pixels have an increased signal, but so do all the pixels leading to the end of the amplifier in the parallel direction.
In addition to the persistence from the bright source, the pixels through which the source is transferred during readout also show persistence, leaving a trail from the source to the edge of the amplifier.
Figure \ref{fig:cuts} shows an example of the spot and the trail and highlights some of their differences.

\subsection{Run 6 Measurements}
Following this discovery in Run 5, we tested and characterized the persistence further in Run 6a and 6b. 
This included using both a flat illuminination to induce persistence as well as utilizing a pinhole filter to create saturated spots to measure the persistence.

\subsubsection{Flat Induced Persistence}
\begin{figure*}[!htp]
  \centering
  % \includegraphics[width=0.48\textwidth, angle=0]{Run_13392_Persistence.png}
  % \includegraphics[width=0.48\textwidth, angle=0]{Persistence_Flat_Example.png}
  \includegraphics[width=0.48\textwidth, angle=0]{Ex_Flat_FP_combined.png}
  \caption{
  Full focal plane images of the counts in the first dark image after a bright exposure for run 13392.
  The left shows the amplifier level average for all science sensors of the subsequent dark.
  The right shows a binned image of the dark image from RubinTV\@.
  % This particular flash has one of the three spots inbetween two amplifiers and you can see the trail starting at the spots and going all the way to the readout of each amplifier.
  }\label{fig:Flat_Induced}
\end{figure*}

Flat illuminated persistence measurements were taken during B and C protocol runs.
The procedure for these measurements were to take a incredibly bright (3--4 times full well) flat followed by 15 second dark images.
These images are then run through both \texttt{$cp\_pipe$} and \texttt{$eo\_pipe$}.
This procedure is designed to induce the persistence, determine the level and then track its decay.
Unfortunately this method does not allow us to measure the trail caused by the persistence as all the pixels are saturated.

\begin{figure*}[!htp]
  \centering
  \includegraphics[width=0.45\textwidth, angle=0]{Run_13392_Persistence.png}
  \includegraphics[width=0.45\textwidth, angle=0]{Run_13410_Persistence_Lower.png}
  \caption{
  Full Focal plane images with amplifier level resolution comparing a subsequent dark image with nominal parallel clock swing voltages (left; Run 13392) 
  and one with a slightly higher parallel clock swing voltage (right; Run 13410).
  % This particular flash has one of the three spots inbetween two amplifiers and you can see the trail starting at the spots and going all the way to the readout of each amplifier.
  }\label{fig:Flat_Induced_Comparison}
\end{figure*}

One of the concerning discoveries with these measurements were the level of persistence was reaching as high as 10 ADU.
Figure \ref{fig:Flat_Induced} shows an example of a focal plane plot displaying the per amp value of the first dark image after the flash as well as the actual dark image.
This figure shows that not only are there large variations between E2V detectors, but also variations on the amplifier level.

One of the ways to lower or remove the persistence effect is to adjust the parallel swing voltages of the detector.
This was tested and Figure \ref{fig:Flat_Induced_Comparison} comapares the per amp plot between the nominal voltages and the slightly higher parallel clock swing voltages (-9.3 vs -9.0 V).
This change did lower the persistence level but much less than what was expected (only $10\%$).


\subsubsection{Spot Induced Persistence}
During Run 6b, we utilized the pinhole filter to measure the persistence of the E2V sensors.
The pinhole filter is positioned so that a small area of light hits one amplifier on each of the middle detectors on a raft.
This means that we can only measure the persistence of 13 e2V detectors but we are able to get measurements across the focal plane.
For these tests, we started with flashes from around 20 ke-/pixel below full well (~100 ke-/pixel) and ramped to twice full well (200 ke-/pixel).
After each flash, we take 10 dark images with 15 second integration times which we use to measure the persistence.
These images are then overscan and bias corrected.
Figure \ref{fig:Run6_example} shows an example of one of the saturated pixels (spot) and the subsequent dark image with persistence.

\begin{figure*}[!htp]
  \centering
  \includegraphics[width=0.85\textwidth, angle=0]{Run6_ex.png}
  \caption{
  Similar to Figure \ref{fig:ex_persistence_Run5} but with the pinhole filter. 
  This shows the initial flash and the spot (left) and the dark images that followed.
  }\label{fig:Run6_example}
\end{figure*}


Using these images, we are able to measure the persistence induced by the saturated pixels as well as the trail that follows.
We then model the persistence decay using this function:
\begin{equation*}
  \textrm{Counts}= A * \exp(-t/\tau)+c
\end{equation*}
where \textit{Counts} are the average number of counts at the location of the spot in the following dark image, \textit{t} is the time since flash, 15 seconds of which are with the CCDs integrating, \textit{$\tau$} is the decay constant and \textit{A} and \textit{c} are constants.
Figure \ref{fig:Run6_decay} shows an example of the decay of the both the spot and the trail and the corresponding model fit for one detector.

\begin{figure*}[!ht]
  \centering
  \includegraphics[width=0.95\textwidth, angle=0]{Run6_decay_2.png}
  \caption{
  The decay of the persistence with consecutive dark images. 
  The blue points are the measured average counts of the spot (left) and trail (right).
  The red line shows the model fit with the parameters shown in the upper right.
  }\label{fig:Run6_decay}
\end{figure*}

Figure \ref{fig:Pers_Flash_Tau} shows the values for \textit{A} and \textit{$\tau$} for all measured e2v detectors and all exposures that showed persistence. 
This figure shows that while \textit{$\tau$} is roughly the same for all images (between 10--20 seconds), the amplitude of the effect changes.
There also is a relationship between flashtime/the brightest of the spot and the persistence measured signal.

\begin{figure*}[!ht]
  \centering
  \includegraphics[width=0.95\textwidth, angle=0]{Persistence_Flash_Tau.png}
  \caption{
  The relationship between the duration of the LED flash (a proxy for the brightness of a saturated source) and amplitude (left) and tau (right) for the spot.
  Each point represents the measured tau and amplitude values of a single sensor from a single exposure that displayed persistence.
  The amplitude appears to be dependent on the flash time, while tau is invariant and is consistently between 10--20 seconds. 
  }\label{fig:Pers_Flash_Tau}
\end{figure*}

\section{Effect of Persistence on DC2 Data}

Though studies are currently being done to minimize or eliminate persistence amplitude, both through hardware and processing, one of the next steps is to see how much persistence will affect LSST photometric measurements.
To study these effects, we used DESC DC2 data and assume that the persistence will affect all saturated pixels with the characteristics of a \textit{$\tau$} of 15 seconds and an amplitude of 5 ADU\@.
We set \textit{c} to zero in this case as there should be no offset with DC2 data.
These numbers are derived from the pinhole data, which we feel will better represent on-sky data.
As most of the characterization has been with the spot and not the trail, we use 5 different models to determine the impact of the trail: assume the trail carries with it 100\%, 75\%, 50\%, 25\%, and 0\% of the spot persistence level.
We only used one of the e2V detectors at random (detector number 88/R21-S21) for this analysis and then assume that this detector is representative of the other e2V detectors.
We also only utilize single-visit images with 30 second exposures to mimic what the effect would be as images would be coming from the observatory.

\subsection{Affected Pixel Counts}

One of the first studies we did was to look at the number of affected by persistence during a simulated night of observing.
To do this, we utilized the 1000 \texttt{calexp} images from DC2 dataset. 
We sorted these images by exposure number in order to apply the persistence as if it was a normal night of observing.
For each individual image, we first apply a running list of persisted pixels following the equation as described above.
For this list, we only included persistence that had a signal strength greater than 1 ADU, as below this value is reaching the level of the bias noise.
After this, we identify the saturated pixels in the image and mapped these pixels, as well as then creating a trail from the saturated pixels to the readout of the amplifier.
We then repeat this procedure for all 1000 images.
% If the persistence dropped below 1 ADU, the persistence were taken off the list, as this is starting to reach the level of the bias noise.
Table \ref{tab:persis_affected}shows the average number of affected pixels in a single image in each of the trail models.

\begin{table}[h!]
\centering
\begin{tabular}{|lcccc|}
\hline
\% of Spot in Trail & >3 ADU ($10^{-4}$) & >1 ADU ($10^{-4}$) & >0.1 ADU ($10^{-4}$) & >0.01 ADU ($10^{-4}$) \\
\hline
100 & 5.85 & 5.85 & 11.7 & 23.4 \\
75 & 5.85 & 5.85 & 11.7 & 17.5 \\
50 & 0.0216 & 5.85 & 11.7 & 17.5 \\
25 & 0.0216 & 5.85 & 11.7 & 17.5 \\
0 & 0.0216 & 3.91 & 7.81 & 11.7 \\
% Dark &
\hline
\end{tabular}\label{tab:persis_affected}
\caption{Table showing the average about of pixels in persistence-affected regions above certain ADU levels. 
The numbers are then normalized to fraction of total CCD pixels affected.
We compare with different trail levels.
%  taken from a recent EO testing Opsim run.
}  
\end{table}



\subsection{Persistence Induced Flux Effects on Objects}

Another possible metric to gauge the impact of persistence is the object measurements.
For this study, we created a new dataset type called \texttt{$raw\_modified$}. 
This dataset type takes the \texttt{$raw$} file and adds the persistence model as described above utilizing the previous images.
For this study, we utilized a small subset of the DC2 dataset (68 images) and the latest reprocessing using the latest verison of the Data Release Pipeline (DRP).
We then run the \texttt{$raw\_modified$} images through the \texttt{$isr$}, \texttt{$characterizeImage$}, and \texttt{$calibrate$} steps of the DRP\@. 

Figure \ref{fig:ex_persistence} shows an example image of an object comparing the two \texttt{$calexp$} images from \texttt{$raw$} and \texttt{$raw\_modified$}.
While the persistence is small enough that is extremely difficult to find by eye, the small boost in signal can affect photometric measurements of objects.
Using these 68 images, we measured the differences in magnitude of objects centered in the persistence-affected regions.
Figure \ref{fig:affected_objects_100} and \ref{fig:affected_objects_other} show the difference in magnitude using both the an aperture flux and PSF flux for the various trail scenarios.
While this effect can be extreme, it also affects a very small percentage of objects, as seen in Figure \ref{fig:cuml}.

\begin{figure*}[!htp]
  \centering
  \includegraphics[width=0.95\textwidth, angle=0]{Obj_pers.png}
  \caption{
  A zoomed in object that shows the persistence-affected image (left), the original image (middle) 
  and the subtraction between the two to highlight the persistence (right).
  }\label{fig:ex_persistence}
\end{figure*}

\begin{figure*}[!htp]
  \centering
  \includegraphics[width=0.95\textwidth, angle=0]{Trail_100.png}
  \caption{
  The changes of magnitude of objects in persistence-affected regions for all the $100\%$ trail regime. 
  In the blue are the changes in magnitude using the calculated PSF, the changes in magnitude using an aperture (orange), and the average changes in the PSF magnitudes for every 0.5 mag (red)
  Also included is the number of objects affected as well as the fraction of objects affected vs total objects.
  These plots only look at magnitudes up to 25th mag.
  There are sources at magnitudes >25 mag but 25th magnitude is the magnitude limit of single exposures.
  }\label{fig:affected_objects_100}
\end{figure*}

\begin{figure*}[!htp]
  \centering
  \includegraphics[width=0.95\textwidth, angle=0]{Trail_75_0.png}
  \caption{
  Simliar to Figure \ref{fig:affected_objects_100} but with the other four trail regimes.
  }\label{fig:affected_objects_other}
\end{figure*}

\begin{figure*}[!htp]
  \centering
  \includegraphics[width=0.95\textwidth, angle=0]{Cumlative_Persistence_Plot.png}
  \caption{
  Plots showing the cumlation percentage of affected objects when compared to all objects from the 68 images with all the trail scenarios. 
  The left plot highlights the percentage of the most affected objects. 
  The right plot zooms into the area where the cumulation trend levels out.
  }\label{fig:cuml}
\end{figure*}

% Since this effect will not be easily removed, I looked into how persistence will affect LSST like images and measurements on the objects. 
% I adopted the same model as desribed above for the persistence and its trail and added it into subsquent images. 
% Figure ~\ref{fig:ex_persistence} shows an example of a DC2 image and the model persistence applied to it. 
% I then plotted a histogram of the fraction of persistence affected pixels as seen in Figure ~\ref{fig:affected_pixels}. 
% This shows the number of images with the corresponidng fraction of affected pixels. 
% For most images, $<0.2\%$ of pixels are affected in the subsquent images.


% \section{Effect of Persistence on DC2 Objects}

% \begin{figure*}[!htp]
%   \centering
%   \includegraphics[width=0.95\textwidth, angle=0]{Obj_pers.png}
%   \caption{
%   A zoomed in object that shows the persistence affected image (left), the original image (middle) 
%   and the subtraction between the two to highlight the persistencet (right)
%   }\label{fig:obj_persistence}
% \end{figure*}

% To measure the effect that this would have on DC2 objects, I used a small subset of data (around 20 images) from DC2 
% found in this collection \texttt{$`2.2i/runs/test-med-1/w_2023_17/DM-39092'$}. 
% Taking the raw images from this dataset, 
% I ran a similar process to the procedure in measuring the effect on an a suite of DC2 images and added the persistence in. 
% I then saved these images as \texttt{$`raw_modified'$} images and ran these types of images through the DRP pipeline up until the `calexp' image step.
% These images can be found in the collection \texttt{$`u/banovetz/dc2/w_2023_34/raw_persistence_single_value_output'$}

% There were roughly 200 objects in these 20 images that were affected by persistence an example of which can be seen in Figure \ref{fig:obj_persistence}.
% Comparing these objects with and without persistence, 
% I found that those with persistence would have a difference of magnitude of about 0.001-0.01 magnitude for those dimmer than ~19-20 magnitude. 
% Figure \ref{fig:absolute_fractional_flux} shows the fractional magnitude difference between the persistence affected images and the originals
%  as a function of magnitude ordered objects (with 0 being the brightest object).

% \begin{figure*}[!htp]
%   \centering
%   \includegraphics[width=0.95\textwidth, angle=0]{Absolute_Fractional_Magnitude.png}
%   \caption{
%   A zoomed in object that shows the persistence affected image (left), the original image (middle) 
%   and the subtraction between the two to highlight the persistencet (right)
%   }\label{fig:absolute_fractional_flux}
% \end{figure*}

% This can be explained as these objects have an area of roughtly 100 square pixels and the trails caused by the persisence can be 10--20 pixels wide.
% Increase the flux by 100--600 ADU of an object would cause this discrepancy.


\appendix

% Here are the relevant collections for what we used for the DC2 comparison:


% Include all the relevant bib files.
% https://lsst-texmf.lsst.io/lsstdoc.html#bibliographies
% \section{References}\label{sec:bib}
% \renewcommand{\refname}{} % Suppress default Bibliography section
% \bibliography{local,lsst,lsst-dm,refs_ads,refs,books}

% Make sure lsst-texmf/bin/generateAcronyms.py is in your path
% \section{Acronyms}\label{sec:acronyms}
% \input{acronyms.tex}
% If you want glossary uncomment below -- comment out the two lines above
%\printglossaries





\end{document}
